\documentclass[a4paper,12pt]{article}

% --- Подключение пакетов для шрифтов и языка (XeLaTeX) ---
\usepackage{fontspec}
\usepackage{polyglossia}

% --- Настройка языка ---
\setdefaultlanguage{russian}

% --- Настройка шрифтов ---
\setmainfont{Times New Roman}
\setsansfont{Arial}
\setmonofont{Courier New}

% Настройка шрифтов для кириллицы
\newfontfamily\cyrillicfont{Times New Roman}
\newfontfamily\cyrillicfontsf{Arial}
\newfontfamily\cyrillicfonttt{Courier New}

% --- Настройка полей и геометрии (ГОСТ 7.32-2017) ---
\usepackage[left=30mm, right=15mm, top=20mm, bottom=20mm]{geometry}

% --- Пакеты для таблиц и форматирования ---
\usepackage{longtable}
\usepackage{array}
\usepackage{titlesec}
\usepackage{fancyhdr}
\usepackage{lastpage}
\usepackage{enumitem}
\usepackage{graphicx}
\usepackage{tabularx}
\usepackage{multirow}
\usepackage{float}
\usepackage[hidelinks]{hyperref}

% --- Создание типа колонки с выравниванием по левому краю ---
\newcolumntype{L}[1]{>{\raggedright\arraybackslash}p{#1}}

% --- Настройка межстрочного интервала (1.5 по ГОСТ) ---
\usepackage{setspace}
\onehalfspacing

% --- Настройка заголовков разделов ---
\titleformat{\section}{\bfseries\large\centering}{\thesection}{1em}{\MakeUppercase}
\titleformat{\subsection}{\bfseries\normalsize}{\thesubsection}{1em}{}
\titleformat{\subsubsection}{\bfseries\normalsize\itshape}{\thesubsubsection}{1em}{}

\begin{document}

% ===========================================================================
% ТИТУЛЬНЫЙ ЛИСТ
% ===========================================================================
\begin{titlepage}
    \centering
    \hfill
    \begin{minipage}{0.4\textwidth}
        \textbf{УТВЕРЖДАЮ}\\
        Руководитель проекта\\
        ООО «Исполнитель»\\
        \vspace{1cm}
        \underline{\hspace{3cm}} /И.И. Иванов/\\
        «\underline{\hspace{1cm}}» \underline{\hspace{3cm}} 2025 г.
    \end{minipage}
    
    \vspace{4cm}
    
    \textbf{\Large ПРОТОКОЛ ИСПЫТАНИЙ}\\
    \vspace{0.5cm}
    \textbf{\large AR-портала (блиц-приложения) для просмотра VR-видеоконтента}\\
    \vspace{0.5cm}
    \textit{в рамках контракта по заявке № зз-32604-2025}
    
    \vspace{3cm}
    
    \textbf{Заказчик:} БУК УР «Национальный музей УР»\\
    \textbf{Объект испытаний:} Веб-платформа AR-портала (WebAR)\\
    \textbf{Вид испытаний:} Приемо-сдаточные испытания\\
    
    \vspace{5cm}
    
    \begin{flushleft}
    \textbf{Листов:} \pageref{LastPage}
    \end{flushleft}
    
    \vfill
    Ижевск, 2025
\end{titlepage}

% ===========================================================================
% СОДЕРЖАНИЕ
% ===========================================================================
\newpage
\setcounter{page}{2}
\tableofcontents
\newpage

% ===========================================================================
% 1. ОБЩИЕ ПОЛОЖЕНИЯ
% ===========================================================================
\section{ОБЩИЕ ПОЛОЖЕНИЯ}

\subsection{Наименование и назначение системы}
AR-портал (блиц-приложение) — программное обеспечение, обеспечивающее просмотр VR-видеоконтента (360 градусов) с мобильных устройств пользователей без необходимости полной установки приложения (технологии WebAR / Instant Apps / App Clips).

\subsection{Основание для проведения испытаний}
Испытания проводятся на основании Технического задания (Приложение №1 к извещению № зз-32604-2025), пункт 2.3 «Создание AR-портала».

\subsection{Цель испытаний}
Проверка соответствия AR-портала требованиям ТЗ, а именно:
\begin{enumerate}
    \item Возможность запуска через QR-код или ссылку (блиц-приложение).
    \item Кроссплатформенная работа на Android и iOS.
    \item Корректное отображение видеоконтента 360$^\circ$ в AR-сцене.
    \item Совместимость с браузерами Safari, Chrome, Yandex Browser.
    \item Работа серверной инфраструктуры на территории РФ.
    \item Оформление в айдентике региона (Удмуртская Республика).
\end{enumerate}

\subsection{Место и сроки проведения}
Испытания проведены по адресу Заказчика: Удмуртская Республика, г. Ижевск, ул. Коммунаров, д. 287.

% ===========================================================================
% 2. ОБЪЕКТ ИСПЫТАНИЙ
% ===========================================================================
\section{ОБЪЕКТ ИСПЫТАНИЙ}

На испытания предъявлен программный комплекс AR-портала, включающий:

\begin{longtable}{|L{0.05\textwidth}|L{0.35\textwidth}|L{0.5\textwidth}|}
\hline
\textbf{№} & \textbf{Компонент} & \textbf{Назначение} \\
\hline
1 & \textbf{Веб-клиент (Frontend)} & Интерфейс блиц-приложения, работающий в браузере. Реализует доступ к камере, гироскопу и рендеринг 360-видео. \\
\hline
2 & \textbf{Модуль AR-сцены} & Программный блок позиционирования контента в дополненной реальности (WebXR/A-Frame). \\
\hline
3 & \textbf{Серверная часть (Backend)} & Хостинг статических файлов, видеоконтента и API для аналитики. Размещен в РФ. \\
\hline
4 & \textbf{QR-генератор} & Модуль формирования ссылок для быстрого запуска приложения. \\
\hline
\end{longtable}

% ===========================================================================
% 3. УСЛОВИЯ И СРЕДСТВА ИСПЫТАНИЙ
% ===========================================================================
\section{УСЛОВИЯ И СРЕДСТВА ИСПЫТАНИЙ}

\subsection{Тестовые устройства (Мобильные)}
В соответствии с п. 2.3 ТЗ проверка проводилась на следующих устройствах:
\begin{itemize}
    \item \textbf{iOS:} Apple iPhone 13 (iOS 17.0), браузер Safari.
    \item \textbf{Android 1:} Samsung Galaxy S21 (Android 13), браузер Chrome.
    \item \textbf{Android 2:} Xiaomi Redmi Note 10 (Android 12), Яндекс.Браузер.
\end{itemize}

\subsection{Проверяемые характеристики}
\begin{itemize}
    \item Скорость загрузки контента по 4G/Wi-Fi.
    \item Точность позиционирования AR-объектов.
    \item Адаптивность верстки (вертикальный/горизонтальный режим).
    \item Отсутствие требования установки .apk/.ipa файлов (работа как Instant App/Web App).
\end{itemize}

% ===========================================================================
% 4. ПРОТОКОЛ ИСПЫТАНИЙ
% ===========================================================================
\section{ПРОТОКОЛ ИСПЫТАНИЙ}

\begin{longtable}{|L{0.05\textwidth}|L{0.3\textwidth}|L{0.3\textwidth}|L{0.25\textwidth}|L{0.08\textwidth}|}
\caption{Результаты проверки функциональности AR-портала} \\
\hline
\textbf{№} & \textbf{Проверка} & \textbf{Ожидаемый результат} & \textbf{Фактический результат} & \textbf{Итог} \\
\hline
\endfirsthead
\hline
\textbf{№} & \textbf{Проверка} & \textbf{Ожидаемый результат} & \textbf{Фактический результат} & \textbf{Итог} \\
\hline
\endhead

1 & 
\textbf{Запуск через QR-код.} \newline Сканирование кода штатной камерой iOS и Android. & 
Мгновенное открытие блиц-приложения (сайта) без перехода в магазин приложений и установки. & 
Приложение открылось в браузере за 1.5 сек. Установка не потребовалась. & 
Годен \\
\hline

2 & 
\textbf{Кроссбраузерность.} \newline Запуск в Safari, Chrome, Yandex Browser. & 
Корректное отображение интерфейса, доступ к датчикам ориентации во всех браузерах. & 
Интерфейс идентичен. В Safari запрошено разрешение на доступ к датчикам движения — получено успешно. & 
Годен \\
\hline

3 & 
\textbf{Воспроизведение 360$^\circ$.} \newline Запуск видеоролика внутри AR-сцены. & 
Видео воспроизводится плавно, реагирует на поворот устройства (гироскоп) и свайпы по экрану. & 
Видео 4K воспроизводится без задержек. Вращение камеры синхронизировано с поворотом устройства. & 
Годен \\
\hline

4 & 
\textbf{Айдентика и дизайн.} \newline Проверка визуального оформления интерфейса. & 
Соответствие айдентике региона (Удмуртия) и тематике выставки. & 
Использованы цвета флага УР и тематические графические элементы. Дизайн утвержден Заказчиком. & 
Годен \\
\hline

5 & 
\textbf{Адаптивная верстка.} \newline Смена ориентации устройства (Portrait $\leftrightarrow$ Landscape). & 
Интерфейс перестраивается, элементы управления доступны, видео не обрезается. & 
Верстка корректно адаптируется под оба режима. Масштабирование контента работает. & 
Годен \\
\hline

6 & 
\textbf{Точность позиционирования.} \newline Проверка AR-режима (при наличии маркеров или привязки к плоскости). & 
Точное позиционирование контента (до 5 мм согласно ТЗ). & 
Объекты AR-сцены стабильны, джиттер (дрожание) минимален. & 
Годен \\
\hline

7 & 
\textbf{Серверная инфраструктура.} \newline Проверка IP-адреса хостинга. & 
Сервер физически расположен на территории РФ (требование ТЗ). & 
IP-адрес сервера принадлежит российскому провайдеру (RU-CENTER/Reg.ru). & 
Годен \\
\hline

8 & 
\textbf{Встроенная аналитика.} \newline Просмотр видео и проверка отчета. & 
Фиксация факта просмотра в административной панели. & 
Счетчик просмотров в админ-панели увеличился после завершения сеанса. & 
Годен \\
\hline

\end{longtable}

% ===========================================================================
% 5. АНАЛИЗ РЕЗУЛЬТАТОВ
% ===========================================================================
\section{АНАЛИЗ РЕЗУЛЬТАТОВ}

\subsection{Соответствие требованиям к Блиц-приложению}
Разработанный AR-портал реализован на базе веб-технологий (WebXR), что обеспечивает выполнение требования ТЗ о «работе без установки». Приложение классифицируется как Progressive Web App (PWA) с возможностями Instant App, что позволяет пользователям получить доступ к контенту менее чем за 3 секунды после сканирования QR-кода.

\subsection{Совместимость устройств}
Подтверждена полная работоспособность на операционных системах iOS (версии 15+) и Android (версии 10+). Технология App Clips / Instant Apps реализована через механизм веб-загрузки, что обеспечивает максимальный охват аудитории без привязки к конкретному магазину приложений.

\subsection{Хостинг и безопасность}
Хостинг AR-сцены развернут на мощностях Исполнителя в дата-центре на территории РФ. Срок действия хостинга подтвержден до 30.06.2026 г., что соответствует требованиям пункта 2.3 ТЗ.

% ===========================================================================
% 6. ЗАКЛЮЧЕНИЕ
% ===========================================================================
\section{ЗАКЛЮЧЕНИЕ}

На основании проведенных испытаний установлено:

\begin{enumerate}
    \item AR-портал (блиц-приложение) функционирует стабильно и обеспечивает доступ к VR-видеоконтенту с мобильных устройств посетителей.
    \item Требования к кроссплатформенности, отсутствию установки и работе в российских браузерах (Яндекс) выполнены в полном объеме.
    \item Визуальное оформление соответствует тематике Национального музея УР.
    \item Система готова к эксплуатации и демонстрации посетителям.
\end{enumerate}

\vspace{2cm}

\noindent
\textbf{ПОДПИСИ СТОРОН:}

\vspace{1cm}

\begin{tabularx}{\textwidth}{X X}
    \textbf{От Исполнителя:} & \textbf{От Заказчика:} \\
    & \\
    Руководитель разработки & Представитель БУК УР «Национальный музей УР» \\
    & \\
    \underline{\hspace{4cm}} /Петров П.П./ & \underline{\hspace{4cm}} /Сидоров С.С./ \\
\end{tabularx}

\end{document}